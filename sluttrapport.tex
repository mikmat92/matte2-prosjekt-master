% latexmal.tex - Mal beregnet for bruk i INF1080
% Time-stamp: <2012-08-27 11:08:39>
%%%%%%%%%%%%%%%%%%%%%%%%%%%%%%%%%%%%%%%%%%%%%%%%%%%%%%%%%%%%%%%%%%%%%%%%%%%
%
% Dette dokumentet har innstillinger som fungerer på Ifi-serverne.
% Du kan lage en pdf-fil av denne filen med:
%  pdflatex latexmal.tex
%
% Et LaTeX-dokument har to deler. Først skriver du inn ting som
% gjelder hele dokumentet, blant annet hvilke pakker du skal bruke.
% Så kommer selve innholdet, mellom \begin{document} og \end{document}
%
% Husk at tegnene: # $ % ^ & _ { } ~ og \ har spesiell betydning.
%
% For mer hjelp og mer info om hvordan du kan stille inn dokumentet
% er disse sidene bra:
% http://www.mn.uio.no/ifi/tjenester/it/hjelp/latex/
% http://en.wikibooks.org/wiki/LaTeX/
%
%%%%%%%%%%%%%%%%%%%%%%%%%%%%%%%%%%%%%%%%%%%%%%%%%%%%%%%%%%%%%%%%%%%%%%

% 1. Hva slags dokument.
\documentclass[12pt,norsk,a4paper]{article}
\usepackage{float}

\usepackage{hyperref}
\usepackage[all]{hypcap}

\hypersetup{
    colorlinks,
    citecolor=black,
    filecolor=black,
    linkcolor=black,
    urlcolor=blue
}

%\usepackage{apacite}
%\usepackage{mathtools}
\usepackage{amsmath}
% For eget tittelblad: legg ,titlepage til rett etter a4paper
% For tosidig: legg til ,twopage


% 3. Norske tegn og norsk utseende

% 3a. Tekstfiler med norske tegn kan lagres i utf-8 eller iso-latin-1
% (iso-8859-1). På Ifi-maskinene er iso-8859-1/15 standard.
\usepackage[utf8]{inputenc} % evt utf8 i stedet for latin1
%\DeclareUnicodeCharacter{030A}{~}

% Please add the following required packages to your document preamble:
\usepackage[table,xcdraw]{xcolor}
% If you use beamer only pass "xcolor=table" option, i.e. \documentclass[xcolor=table]{beamer}
\usepackage{listings}
%\usepackage[title]{appendix}
% 2. Om dokumentet - brukes blant annet til tittelen
\title{Prosjektoppgave TDAT2002}
\author{Andreas Børsheim, Markus Markussen,\\
		Mikael Mathisen, Eirik Nielsen}
\date{\today}


% 3b. For norsk orddeling og dato
\usepackage[norsk]{babel}
\usepackage[norsk]{isodate}

% 3c. For norske tegn
\usepackage[T1]{fontenc}

% 3d. For parallelle avsnitt (fint i INF1080, men ikke så fint i vanlige artikler)
\usepackage{parskip}

% 3e. For å formattere URL-er.
%\usepackage{url}

% 4. Skrifttype og symboler
\usepackage[small,euler-digits]{eulervm}
%\usepackage[bitstream-charter]{mathdesign}
\usepackage{color}

% 5. For litt mindre marger enn standard i LaTeX
\usepackage{a4wide}
\usepackage[top=2cm, margin=1.5cm]{geometry}

% 6. Figurer og bilder

% 6a. For å bryte tekst rundt en figur
%\usepackage{wrapfig}

% 6b. For bilder - med denne pakken kan du legge inn bilder og
% illustrasjoner slik:
%                  \includegraphics[width=.5\textwidth]{bilde.pdf}
\usepackage{graphicx}
\graphicspath{ {img/} }

% 7. For grafer og slikt
%\usepackage{tikz}
%\usetikzlibrary{trees}
% Det fins mange tikz-bibliotek, det er nesten ingen grenser for hva
% du kan lage med dette. Se http://www.texample.net/tikz/
%\usepackage[]{mcode}
% 8. Egendefinerte kommandoer kan du legge inn slik:
\newcommand{\tuple}[1]{\ensuremath{\langle #1\rangle}}
\newcommand{\set}[1]{\ensuremath{\{#1\}}}
\newcommand{\imp}{\ensuremath{\rightarrow}}
\newcommand{\M}{\ensuremath{\mathcal{M}}}
\newcommand{\AF}{edge from parent[draw=none]}
\newcommand{\NODE}[1]{\{node \{\ensuremath{#1}\}\}}
\newcommand\numberthis{\addtocounter{equation}{1}\tag{\theequation}}
% http://en.wikibooks.org/wiki/LaTeX/Customizing_LaTeX#New_commands
\usepackage{textcomp}
\lstset{extendedchars=\true}
\lstset{inputencoding=latin1}

%%%%%%%%%%%%%%%%%%%%%%%%%%%%%%%%%%%%%%%%%%%%%%%%%%%%%%%%%%%%%%%%%%%%%%%%%%%
% Selve innholdet:
\begin{document}

% a. Lager en tittel på dokumentet.
\maketitle
% a. For å få bokstavnummerering på 'subsection's
%\renewcommand{\thesubsection}{(\alph{subsection})}
%\vspace{20mm}

\newpage
\textbf{Revisjonshistorie} \\
\begin{tabular}{|l|l|l|l|}
    \hline
    \textbf{Dato} & \textbf{Versjon} & \textbf{Beskrivelse} & \textbf{Forfatter} \\
    \hline
    4 Mars 2016 & 0.1 & Første versjon & ESN \\
    \hline
    3 April 2016 & 0.2 & Andre versjon & ESN \\
    \hline
\end{tabular}
\newpage
\tableofcontents
%\listoffigures
%\listoftables
\newpage

\include{tex/innledning}
\newpage
\section{Teori og metode}
\label{sec:teori}
\subsection{Euler-Bernoullibjelken}
Euler-Bernoullibjelken er en modell som beskriver hvordan materialer bøyer seg under belastning. Den vertikale forskyvningen \textit{y(x), 0 $\leq$ x $\leq$ L} for en bjelke av lengde \textit{L}, tilfredsstiller differensialligningen
\begin{quote}
\begin{equation}\label{eq:eulerbernoulli}
EIy'''' = f(x)
\end{equation}
\end{quote}
hvor \textit{E} er materialets Youngmodulus, \textit{I} er bjelkens arealmoment. Disse er konstante langs bjelkens lengderetning. Ligningens høyreside \textit{f(x)} er den påførte belastningen, inkludert bjelkens egenvekt. \cite[s.~102]{mattebok} \\

For å løse dette numerisk, ønsker vi å finne en diskretisering av \textit{y\textsuperscript{(4)}(x)}, og betrakte bjelken som unionen av segmenter med lengde \textit{h}. Vi beviser i \ref{sec:oppg1} at den diskretiserte formelen for \textit{y\textsuperscript{(4)}(x)} er:
\begin{quote}
\begin{equation} \label{eq:y''''}
y^{(4)}(x) = \frac{y(x+2h)-4y(x-h)+6f(x)-4y(x+h)+y(x+2h)}{h^4}
\end{equation}
\end{quote}
med et feilledd proporsjonalt med \textit{h\textsuperscript{2}}.\\
Vi deler bjelken inn i \textit{n > 0} like deler, slik at \textit{h = L/n}. La så \textit{x\textsubscript{i} = i $\cdot$ h, i = 0, 1, 2, ... , n}. Da har vi at \textit{h = x\textsubscript{i} - x\textsubscript{i-1}}, for \textit{i = 1, 2,..., n}. La også \textit{y\textsubscript{i} = y(x\textsubscript{i})}. Setter vi dette inn i ligning (\ref{eq:y''''}), får vi

\begin{align*}
y^{(4)}(x_i)& = \frac{y(x_i-2(x_i-x_{i-1}))-4y(x_i-(x_i-x_{i-1}))+6y(x_i)-4y(x_i+(x_i-x_{i-1}))+y(x_i+2(x_i-x_{i-1}))}{h^4} \\
&= \frac{y(-x_i+2x_{i-1})-4y(x_{i-1})+6y(x_i)-4y(2x_i-x_{i-1})+y(3x_i-x_{i-1})}{h^4} \\
&= \frac{y(x_{i-2})-4y(x_{i-1})+6y(x_{i-1})-4y(x_{i+1})+y(x_{i+2})}{h^4} \\
&= \frac{y_{i-2} - 4y_{i-1} + 6y_i - 4y_{i+1} + y_{i+2}}{h^4} \numberthis \label{eq:y4discr}
\end{align*}

Til slutt setter vi sammen ligning (\ref{eq:y4discr}) og ligning (\ref{eq:eulerbernoulli}):
\begin{center}
\begin{gather*}
EIy^{4}(x_i) = f(x_i) \\
EI \frac{y_{i-2} - 4y_{i-1} + 6y_i - 4y_{i+1} + y_{i+2}}{h^4} = f(x_i) \\
y_{i-2} - 4y_{i-1} + 6y_i - 4y_{i+1} + y_{i+2} = \frac{h^4}{EI}f(x_i) \numberthis \label{eq:eulerbernoullidiscrete}
\end{gather*}
\end{center}

Dette gjør oss i stand til å lage \textit{n} ligninger for de \textit{n} utkjente \textit{y\textsubscript{1}, ... , y\textsubscript{n}}, ved hjelp av en matriseligning på formen \textit{A\textbf{y} = \textbf{f}}, hvor \textit{\textbf{f}=}[\textit{f(x\textsubscript{1}), f(x\textsubscript{2}), ... , f(x\textsubscript{n})}]\textsuperscript{\textit{T}}, og koeffisientmatrisa \textit{A} er koeffisientene til venstresiden av ligning (\ref{eq:eulerbernoullidiscrete}).\\

Vi ser at ligning (\ref{eq:eulerbernoullidiscrete}) er problematisk i endene av bjelken, hvor \textit{y\textsubscript{i-2}} og \textit{y\textsubscript{i-1}} eller \textit{y\textsubscript{i+2}} og \textit{y\textsubscript{i+1}} ikke er definert. I denne oppgaven skal vi ta for oss en bjelke som er fastmontert i den ene enden og fri i den andre, som et stupebrett. For en slik bjelke er randtilstandene slik at
\begin{equation*}
y(0) = y'(0) = y''(L) = y'''(L) = 0
\end{equation*}
Altså har vi at \textit{y\textsubscript{0}=0}, men \textit{y\textsubscript{1}} vil være avhengig av den udefinerte \textit{y\textsubscript{-1}}. I \ref{sec:oppg1} beviser vi at en annen approksimasjon av \textit{y\textsuperscript{(4)}(x)} er
\begin{equation} \label{eq:y4xapprox2}
y^{(4)}(x) = \frac{16y(x) - 9y(x+h) +\frac{8}{3}y(x+2h)-\frac{1}{4}y(x+3h)}{h^4} + O(h^2)
\end{equation}
som er gyldig når \textit{y(x\textsubscript{0}) = y'(x\textsubscript{0}) = 0}. Vi gir denne ligningen samme behandling som (\ref{eq:y''''}) og setter inn \textit{x = x\textsubscript{1}, h = x\textsubscript{1} - x\textsubscript{0}}, og ender da opp med uttrykket
\begin{equation}\label{eq:y4x1}
y^{(4)}(x_1) = \frac{16y_1 - 9y_2 +\frac{8}{3}y_3-\frac{1}{4}y_4}{h^4}
\end{equation}
Vi setter dette inn i ligning (\ref{eq:eulerbernoulli}), og får ligningen for \textit{y\textsubscript{1}}:
\begin{equation} \label{eq:eulerbernoullix1}
16y_1 - 9y_2 + \frac{8}{3}y_3 - \frac{1}{4}y_4 = \frac{h^4}{EI}f(x_1)
\end{equation}

For å finne et uttrykk i for avbøyningen i den frie enden av bjelken, trenger vi igjen nye approksimasjoner av \textit{y\textsuperscript{(4)}(x)}, fordi ligning (\ref{eq:y4xapprox2}) er avhengig av \textit{y\textsubscript{i+1}} og \textit{y\textsubscript{i+2}}, som åpenbart ikke er definert for \textit{x\textsubscript{n-1}} og  \textit{x\textsubscript{n}}. I \ref{sec:oppg1} beviser vi også at approksimasjonene 
\begin{equation*}
y^{(4)}(x_{n-1}) = \frac{-28y_n + 72y_{n-1} -60y_{n-2}+16y_{n-3}}{17h^4} + O(h^2)
\\
y^{(4)}(x_{n}) = \frac{72y_n -156y_{n-1} +96y_{n-2}-12y_{n-3}}{17h^4} + O(h^2)
\end{equation*}
er gyldige når \textit{y''(x\textsubscript{n}) = y'''(x\textsubscript{n}) = 0}.\\

Nå har vi altså \textit{n} ligninger for de \textit{n} ukjente \textit{y\textsubscript{n}}:
\begin{align*}
&16y_1 - 9y_2 + \frac{8}{3}y_3 - \frac{1}{4}y_4 = \frac{h^4}{EI}f(x_1) \\
&y_{i-2} - 4y_{i-1} + 6y_i - 4y_{i+1} + y_{i+2} = \frac{h^4}{EI}f(x_i),\quad i = 1, 2,..., n-2 \\
&\frac{-28y_n + 72y_{n-1} -60y_{n-2}+16y_{n-3}}{17} = \frac{h^4}{EI}f(x_{n-1}) \\
&\frac{72y_n -156y_{n-1} +96y_{n-2}-12y_{n-3}}{17} = \frac{h^4}{EI}f(x_{n})\\
\end{align*}

Dette kan skrives mer kompakt som en matriseligning:
\begin{equation}
\begin{bmatrix}
16	& -9	& \frac{8}{3}	& -\frac{1}{4}	&	&	&	&	&\\
-4	& 6		& -4			& 1				&	&	&	&	&\\
1	& -4	& 6				& -4			& 1	&	&	&	&\\
	& 1		& -4			& 6				&-4	& 1	&	&	&\\
	&  & \ddots		&\ddots			&\ddots	&\ddots	&\ddots	&\\
	&		& 1	& -4	& 6				& -4			& 1	&\\
	&		&	& 1	& -4	& 6				& -4			& 1\\
	&	&	&	&	&\frac{-12}{17}	&\frac{96}{17}	&\frac{-156}{17}	&\frac{72}{17}\\
	&	&	&	&	&\frac{16}{17}	&\frac{-60}{17}	&\frac{72}{17}	&\frac{-28}{17}\\
\end{bmatrix}
\begin{bmatrix}
y_1\\
y_2\\
\vdots\\
\\
\vdots\\
\\
\vdots\\
y_{n-1}\\
y_n\\
\end{bmatrix} =
\begin{bmatrix}
f(x_1)\\
f(x_2)\\
\vdots\\
\\
\vdots\\
\\
\vdots\\
f(x_{n-1})\\
f(x_n)\\
\end{bmatrix}
\end{equation}
\newpage
\section{Oppgaver}
\subsection{Oppgave 1}\label{sec:oppg1}
\subsubsection{5.1.21}
Vi benytter oss av Taylor's teorem:
\begin{quote}
Dersom \textit{f(x)} er \textit{k} ganger deriverbar på intervallet [\textit{x}, \textit{x\textsubscript{0}}], finnes det et tall \textit{c} slik at
\begin{equation} \label{eq:taylor}
f(x) = \sum_{n=0}^k	(\frac{f^{(k)}(x_{0}}{n!} (x-x_{0})^{k}) + \frac{f^{(k+1)}(c)}{(k+1)!}(x-x_{0})^{(k+1)}
\end{equation}
\end{quote}

La \textit{x = x $\pm$ h} og \textit{x\textsubscript{0} = x}. Taylorpolynomet av 5. grad blir da
\begin{quote}
\begin{equation} \label{eq:x+h}
f(x+h) = f(x) + f'(x)h + \frac{f''(x)}{2}h^2 + \frac{f'''(x)}{6}h^3 + \frac{f^{(4)}(x)}{24}h^4 + \frac{f^{(5)}(x)}{120}h^5 + \frac{f^{(6)}(c)}{720}h^6
\end{equation}
\begin{equation} \label{eq:x-h}
f(x-h) = f(x) - f'(x)h + \frac{f''(x)}{2}h^2 - \frac{f'''(x)}{6}h^3 - \frac{f^{(4)}(x)}{24}h^4 - \frac{f^{(5)}(x)}{120}h^5 + \frac{f^{(6)}(c)}{720}h^6
\end{equation}
\end{quote}

Vi ønsker å kvitte oss med alle deriverte utenom fjerdederiverte, så vi evaluerer (\ref{eq:x+h}) + (\ref{eq:x-h}) og får
\begin{quote}
\begin{equation} \label{eq:x+-h-sum}
f(x+h)+f(x-h) = 2f(x) + f''(x)h^2 + \frac{f^{(4)}(x)}{12}h^4 + \frac{f^{(6)}(c)}{360}h^6
\end{equation}
\end{quote}

Nå ser vi at vo har kvittet oss med alle deriverte utenom fjerde og andre grad. Vi utvikler så Taylorpolynomet av 5. grad med \textit{x = x $\pm$ 2h} og \textit{x\textsubscript{0} = x}:

\begin{quote}
\begin{equation} \label{eq:x+2h}
f(x+2h) = f(x) + f'(x)2h + \frac{f''(x)}{2}4h^2 + \frac{f'''(x)}{6}8h^3 + \frac{f^{(4)}(x)}{24}16h^4 + \frac{f^{(5)}(x)}{120}32h^5 + \frac{f^{(6)}(c)}{720}64h^6
\end{equation}
\begin{equation} \label{eq:x-2h}
f(x-2h) = f(x) - f'(x)2h + \frac{f''(x)}{2}4h^2 - \frac{f'''(x)}{6}8h^3 + \frac{f^{(4)}(x)}{24}16h^4 - \frac{f^{(5)}(x)}{120}32h^5 + \frac{f^{(6)}(c)}{720}64h^6
\end{equation}
\end{quote}

På samme måte som vi fikk ligning (\ref{eq:x+-h-sum}), evaluerer vi (\ref{eq:x+2h})+(\ref{eq:x-2h}):
\begin{quote}
\begin{equation} \label{eq:x+-2h-sum}
f(x+2h)+f(x-2h) = 2f(x) + f''(x)4h^2 + \frac{f^{(4)}(x)}{12}16h^4 + \frac{f^{(6)}(c)}{360}64h^6
\end{equation}
\end{quote}

For å eliminere andrederivertleddet, evaluerer vi 4 $\cdot$ (\ref{eq:x+-h-sum}) - (\ref{eq:x+-2h-sum}):
\begin{quote}
\begin{equation} \label{eq:finalSum}
4f(x+h) + 4f(x-h) - f(x+2h) - f(x-2h) = 6f(x) - h^4f^{(4)}(x) - \frac{f^{(6)}(c)}{6}h^6
\end{equation}
\end{quote}

Vi løser så denne med hensyn på \textit{f\textsuperscript{(4)}(x)}:
\begin{quote}
\begin{equation}
f^{(4)}(x) = \frac{f(x+2h)-4f(x-h)+6f(x)-4f(x+h)+f(x+2h)-\frac{f^{(6)}(c)}{6}h^2}{h^4}
\end{equation}
\end{quote}

Ser at restleddet $\frac{f^{(6)}(c)}{6}h^2 \in O(h^2)$, dvs
\begin{quote}
\begin{equation} \label{eq:fjerdederivert}
f^{(4)}(x) = \frac{f(x+2h)-4f(x-h)+6f(x)-4f(x+h)+f(x+2h)}{h^4} + O(h^2)
\end{equation}
\end{quote}

\subsubsection{5.1.22a}

Skal vise at hvis $f(x)=f'(x)=0$, så er:
\begin{quote}
\begin{equation}
f^{(4)}(x+h) = \frac{16f(x+h)-9f(x+2h)+\frac{8}{3}f(x+3h)-\frac{1}{4}f(x+4h)}{h^4}=O(h^2)
\end{equation}
\end{quote}

Har fra oppgave 5.1.21 at:

\begin{quote}
\begin{equation}
f^{(4)}(x)=\frac{f(x-2h)-4f(x-h)+6f(x)-4f(x+h)+f(x+2h)}{h^4}+O(h^2)
\end{equation}
\end{quote}

Gjør om uttrykket ved å sette inn h, og får:

\begin{quote}
\begin{equation}
f^{(4)}(x+h)=\frac{f(x-2h+h)-4f(x-h+h)+6f(x+h)-4f(x+h+h)+f(x+2h+h)}{h^4}+O(h^2)
\end{equation}
\end{quote}
\begin{quote}
\begin{equation}
f^{(4)}(x+h)=\frac{f(x-h)-4f(x)+6f(x+h)-4f(x+2h)+f(x+3h)}{h^4}+O(h^2)
\end{equation}
\end{quote} 

Setter inn dette for $f^{(4)}(x+h)$ i likning 1, og får:

\begin{quote}
\begin{equation}
\frac{f(x-h)-4f(x)+6f(x+h)-4f(x+2h)+f(x+3h)}{h^4}+O(h^2)= \frac {16f(x+h)-9f(x+2h)+\frac{8}{3}f(x+3h)-\frac{1}{4}f(x+4h)}{h^4}=O(h^2) 
\end{equation}
\end{quote}

\begin{quote}
\begin{equation}
\frac{-4(x)+f(x+h)+6f(x+h)-16f(x+h)-4f(x+2h)+9f(x+2h)+f(x+3h)-\frac{8}{3}f(x+3h)-\frac{1}{4}f(x+4h)+O(h^2}{h^4}
\end{equation}
\end{quote}
\begin{quote}
\begin{equation}
\frac{-4(x)+f(x+h)-10f(x+h)+5f(x+2h)-\frac{5}{3}f(x+3h)+\frac{1}{4}f(x+4h)+O(h^2}{h^4}=O(h^2)
\end{equation}
\end{quote}
Vi har også fra 5.1.21 at hvis $f(x)=f'(x)=0$ så er:
\begin{quote}
\begin{equation}
f(x-h)-10f(x+h)+5f(x+2h)-\frac{5}{3}f(x+3h)+\frac{1}{4}f(x+4h)=O(h^6)
\end{equation}
\end{quote}

Ser at dette stemmer med det vi allerede har, setter inn og får:

\begin{quote}
\begin{equation}
\frac{-4(x)+O(h^6)+O(h^2)}{h^4}=O(h^2)
\end{equation}
\end{quote}

\begin{quote}
\begin{equation}
0+O(h^{(6-4)}+O(h^2)=O(h^2)
\end{equation}

\begin{equation}
O(h^2)=O(h^2)
\end{equation}
\end{quote}

\subsection{Oppgave 6 a}\label{sec:oppg6}
Vi skal legge til en sinusformet haug på bjelken. Det betyr at vi legger til en funksjon 
\begin{equation} \label{eq:sinuslast}
s(x) = -pgsin\frac{\pi}{L}x
\end{equation}
til kraftdelen til \textit{f(x)}

Vi skal vise at \begin{equation}
y(x) = \frac{f}{24EI}x^{2}(x^2-4Lx+6L^2)-\frac{gpL}{EI\pi}(\frac{L^3}{\pi^3}sin\frac{\pi}{L}x-\frac{x^3}{6}+\frac{Lx^2}{2}-\frac{L^2x}{\pi^2})
\end{equation}
tilfredsstiller Euler-Bernoulli-likningen og randbetingelsene for en bhelke som er festet i den ene enden og fri i den andre: \begin{equation} \label{eq:randbetingelserBjelke}
y(0)=y'(0)=y''(L)=y'''(L)=0
\end{equation}

Setter vi sammen ligning (\ref{eq:eulerbernoulli}) og ligning (\ref{eq:sinuslast}) 
Dette gir oss likningen 
\begin{equation} \label{eq:sinusLikning}
EIy''''=-pgsin\frac{\pi}{L}x
\end{equation}

For å finne \textit{y(x)} må vi integrere likning (\ref{eq:sinusLikning}) fire ganger. Dette gir oss:

\begin{equation}
EIy'''(x)=\frac{L}{\pi}pgcos(\frac{\pi}{L}x)+C_1
\end{equation}
\begin{equation}
EIy''(x)=\frac{L^2}{\pi^2}pgsin(\frac{\pi}{L}x)+C_1x+C_2
\end{equation}
\begin{equation}
EIy'(x)=-\frac{L^3}{\pi^3}pgcos(\frac{\pi}{L}x)+\frac{1}{2}C_1x^2+C_2x+C_3
\end{equation}
\begin{equation}
EIy(x)=-\frac{L^4}{\pi^4}pgsin(\frac{\pi}{L}x)+\frac{1}{3}C_1x^3+\frac{1}{2}C_2x^2+C_3x+C_4)
\end{equation}

Ut fra randbetingelsene i ligning (\ref{eq:randbetingelserBjelke}) gir dette oss følgende:

\begin{equation}
y''''(L)=0 \rightarrow \frac{L}{EI\pi}pgcos(\pi)+C_1 \rightarrow \frac{L}{EI\pi}pg=C_1
\end{equation}

\begin{equation}
y'''(L)=0 \rightarrow \frac{L^2}{\pi^2}pgsin(\pi)+C_1x+C_2 \rightarrow C_2=-\frac{pgL^2}{EI\pi}
\end{equation}

\begin{equation}
y''(0)=0 \rightarrow -\frac{L^3}{\pi^3}pgcos(\pi)+C_3 \rightarrow C_3=\frac{L^3pg}{\pi^3}
\end{equation}

\begin{equation}
y'(0)=0 \rightarrow C_4=0
\end{equation}

Dette gir oss følgende ligning for \textit{y(x)}:

\begin{equation}
y(x)=\frac{1}{EI}(-\frac{L^4}{\pi^4}pgsin(\frac{\pi}{L}x)+\frac{L}{6\pi}pgx^3-\frac{L^2}{2\pi}pgx^2+\frac{L^3}{\pi^3}pgx)
\end{equation}

Vi ser at alle leddene har \textit{pgL} og $\frac{1}{\pi}$ til felles så vi trekker dette på utsiden av parentesen. I tillegg vil vi ha \textit{y(x)} alene på venstre side. 

\begin{equation}
y(x)=-\frac{pgL}{EI\pi}(\frac{L^3}{\pi^3}sin(\frac{\pi}{L}x)-\frac{x^3}{6}+\frac{1}{2}Lx^2+\frac{L^2}{\pi^2}x)
\end{equation}

Denne delen legges til bak det vi har vist i oppgave 4 a) og vi får:

\begin{equation}
y(x) = \frac{f}{24EI}x^{2}(x^2-4Lx+6L^2)-\frac{gpL}{EI\pi}(\frac{L^3}{\pi^3}sin\frac{\pi}{L}x-\frac{x^3}{6}+\frac{Lx^2}{2}-\frac{L^2x}{\pi^2})
\end{equation}

Hvilket skulle vises.

\subsection{Oppgave 4 a}\label{sec:oppg4}

Den korrekte løsningen av likningen med konstant \textit{f(x)=f} er

\begin{equation}
y(x)=(\frac{f}{24EI})x^2(x^2-4Lx+6L^2)
\end{equation}

Dette skal vi vise ved å derivere 4 ganger.

Først skriver vi om likningen

\begin{equation}
y(x)=\frac{fx^2(x^2-4Lx+6L^2}{24EI}
\end{equation}

Løser opp parentesen og får:

\begin{equation}
y(x)=\frac{fx^4-4fLx^3+6fL^2x^2}{24EI}
\end{equation}

Så deriverer vi fortløpende:

\begin{equation}
y'(x)=\frac{4fx^3-12fLx^2+12fL^2x}{24EI} = \frac{4(fx^3-fLx^2+3fL^2x}{24EI}
\end{equation}
\begin{equation}
\frac{fx^3-3fLx^2+3fL^2x}{6EI}
\end{equation}

\begin{equation}
y''(x) = \frac{3fx^2-6fLx+3fL^2}{6EI} = \frac{3(fx^2-2fLx+fL^2}{2EI}
\end{equation}

\begin{equation}
y'''(x) = \frac{2fx-2fL}{2EI} = \frac{fx-fL}{EI}
\end{equation}

\begin{equation}
y''''(x) = \frac{f}{EI} = \frac{f(x)}{EI}
\end{equation}

Hvilket skulle vises.

\subsection{Oppgave 4 b}\label{sec:oppg4}
Som gitt i oppgaveteksten har utledningen av formlene for de deriverte brukt Taylors formel med feilledd:
\begin{equation}
\frac{y^{(6)}*(c)}{6!} * h^6
\end{equation}

Vi vet fra oppgave 4a følgende:
\begin{equation}
y''''(x) = \frac{f}{EI}
\end{equation}
\begin{equation}
y^{(5)}(x) = 0
\end{equation}
\begin{equation}
y^{(6)}(x) = 0
\end{equation}
Vi vet at både $y^{(5)} $ og $y^{(6)} $ er lik null, dermed er løsningen eksakt.


\newpage

\appendix
\section{Oppgave 2, Matlabkode}\label{app:oppg2}
\lstinputlisting{matlab/lagA.m}
\section{Oppgave 3, Matlabkode}\label{app:oppg3}
\lstinputlisting{matlab/eulerbernoulli.m}
\lstinputlisting{matlab/Oppgave3.m}
\section{Oppgave 5, Matlabkode}\label{app:oppg5}
\lstinputlisting{matlab/Oppgave5.m}
\bibliographystyle{abbrv}
\bibliography{bib/refs}


\end{document}
