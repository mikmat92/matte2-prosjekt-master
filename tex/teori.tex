\section{Teori og metode}
\label{sec:teori}
\subsection{Euler-Bernoullibjelken}
Euler-Bernoullibjelken er en modell som beskriver hvordan materialer bøyer seg under belastning. Den vertikale forskyvningen \textit{y(x), 0 $\leq$ x $\leq$ L} for en bjelke av lengde \textit{L}, tilfredsstiller differensialligningen
\begin{quote}
\begin{equation}\label{eq:eulerbernoulli}
EIy'''' = f(x)
\end{equation}
\end{quote}
hvor \textit{E} er materialets Youngmodulus, \textit{I} er bjelkens arealmoment. Disse er konstante langs bjelkens lengderetning. Ligningens høyreside \textit{f(x)} er den påførte belastningen, inkludert bjelkens egenvekt. \cite[s.~102]{mattebok} \\

For å løse dette numerisk, ønsker vi å finne en diskretisering av \textit{y\textsuperscript{(4)}(x)}, og betrakte bjelken som unionen av segmenter med lengde \textit{h}. Vi beviser i \ref{sec:oppg1} at den diskretiserte formelen for \textit{y\textsuperscript{(4)}(x)} er:
\begin{quote}
\begin{equation} \label{eq:y''''}
y^{(4)}(x) = \frac{y(x+2h)-4y(x-h)+6f(x)-4y(x+h)+y(x+2h)}{h^4}
\end{equation}
\end{quote}
med et feilledd proporsjonalt med \textit{h\textsuperscript{2}}.\\
Vi deler bjelken inn i \textit{n > 0} like deler, slik at \textit{h = L/n}. La så \textit{x\textsubscript{i} = i $\cdot$ h, i = 0, 1, 2, ... , n}. Da har vi at \textit{h = x\textsubscript{i} - x\textsubscript{i-1}}, for \textit{i = 1, 2,..., n}. La også \textit{y\textsubscript{i} = y(x\textsubscript{i})}. Setter vi dette inn i ligning (\ref{eq:y''''}), får vi

\begin{align*}
y^{(4)}(x_i)& = \frac{y(x_i-2(x_i-x_{i-1}))-4y(x_i-(x_i-x_{i-1}))+6y(x_i)-4y(x_i+(x_i-x_{i-1}))+y(x_i+2(x_i-x_{i-1}))}{h^4} \\
&= \frac{y(-x_i+2x_{i-1})-4y(x_{i-1})+6y(x_i)-4y(2x_i-x_{i-1})+y(3x_i-x_{i-1})}{h^4} \\
&= \frac{y(x_{i-2})-4y(x_{i-1})+6y(x_{i-1})-4y(x_{i+1})+y(x_{i+2})}{h^4} \\
&= \frac{y_{i-2} - 4y_{i-1} + 6y_i - 4y_{i+1} + y_{i+2}}{h^4} \numberthis \label{eq:y4discr}
\end{align*}

Til slutt setter vi sammen ligning (\ref{eq:y4discr}) og ligning (\ref{eq:eulerbernoulli}):
\begin{center}
\begin{gather*}
EIy^{4}(x_i) = f(x_i) \\
EI \frac{y_{i-2} - 4y_{i-1} + 6y_i - 4y_{i+1} + y_{i+2}}{h^4} = f(x_i) \\
y_{i-2} - 4y_{i-1} + 6y_i - 4y_{i+1} + y_{i+2} = \frac{h^4}{EI}f(x_i) \numberthis \label{eq:eulerbernoullidiscrete}
\end{gather*}
\end{center}

Dette gjør oss i stand til å lage \textit{n} ligninger for de \textit{n} utkjente \textit{y\textsubscript{1}, ... , y\textsubscript{n}}, ved hjelp av en matriseligning på formen \textit{A\textbf{y} = \textbf{f}}, hvor \textit{\textbf{f}=}[\textit{f(x\textsubscript{1}), f(x\textsubscript{2}), ... , f(x\textsubscript{n})}]\textsuperscript{\textit{T}}, og koeffisientmatrisa \textit{A} er koeffisientene til venstresiden av ligning (\ref{eq:eulerbernoullidiscrete}).\\

Vi ser at ligning (\ref{eq:eulerbernoullidiscrete}) er problematisk i endene av bjelken, hvor \textit{y\textsubscript{i-2}} og \textit{y\textsubscript{i-1}} eller \textit{y\textsubscript{i+2}} og \textit{y\textsubscript{i+1}} ikke er definert. I denne oppgaven skal vi ta for oss en bjelke som er fastmontert i den ene enden og fri i den andre, som et stupebrett. For en slik bjelke er randtilstandene slik at
\begin{equation*}
y(0) = y'(0) = y''(L) = y'''(L) = 0
\end{equation*}
Altså har vi at \textit{y\textsubscript{0}=0}, men \textit{y\textsubscript{1}} vil være avhengig av den udefinerte \textit{y\textsubscript{-1}}. I \ref{sec:oppg1} beviser vi at en annen approksimasjon av \textit{y\textsuperscript{(4)}(x)} er
\begin{equation} \label{eq:y4xapprox2}
y^{(4)}(x) = \frac{16y(x) - 9y(x+h) +\frac{8}{3}y(x+2h)-\frac{1}{4}y(x+3h)}{h^4} + O(h^2)
\end{equation}
som er gyldig når \textit{y(x\textsubscript{0}) = y'(x\textsubscript{0}) = 0}. Vi gir denne ligningen samme behandling som (\ref{eq:y''''}) og setter inn \textit{x = x\textsubscript{1}, h = x\textsubscript{1} - x\textsubscript{0}}, og ender da opp med uttrykket
\begin{equation}\label{eq:y4x1}
y^{(4)}(x_1) = \frac{16y_1 - 9y_2 +\frac{8}{3}y_3-\frac{1}{4}y_4}{h^4}
\end{equation}
Vi setter dette inn i ligning (\ref{eq:eulerbernoulli}), og får ligningen for \textit{y\textsubscript{1}}:
\begin{equation} \label{eq:eulerbernoullix1}
16y_1 - 9y_2 + \frac{8}{3}y_3 - \frac{1}{4}y_4 = \frac{h^4}{EI}f(x_1)
\end{equation}

For å finne et uttrykk i for avbøyningen i den frie enden av bjelken, trenger vi igjen nye approksimasjoner av \textit{y\textsuperscript{(4)}(x)}, fordi ligning (\ref{eq:y4xapprox2}) er avhengig av \textit{y\textsubscript{i+1}} og \textit{y\textsubscript{i+2}}, som åpenbart ikke er definert for \textit{x\textsubscript{n-1}} og  \textit{x\textsubscript{n}}. I \ref{sec:oppg1} beviser vi også at approksimasjonene 
\begin{equation*}
y^{(4)}(x_{n-1}) = \frac{-28y_n + 72y_{n-1} -60y_{n-2}+16y_{n-3}}{17h^4} + O(h^2)
\\
y^{(4)}(x_{n}) = \frac{72y_n -156y_{n-1} +96y_{n-2}-12y_{n-3}}{17h^4} + O(h^2)
\end{equation*}
er gyldige når \textit{y''(x\textsubscript{n}) = y'''(x\textsubscript{n}) = 0}.\\

Nå har vi altså \textit{n} ligninger for de \textit{n} ukjente \textit{y\textsubscript{n}}:
\begin{align*}
&16y_1 - 9y_2 + \frac{8}{3}y_3 - \frac{1}{4}y_4 = \frac{h^4}{EI}f(x_1) \\
&y_{i-2} - 4y_{i-1} + 6y_i - 4y_{i+1} + y_{i+2} = \frac{h^4}{EI}f(x_i),\quad i = 1, 2,..., n-2 \\
&\frac{-28y_n + 72y_{n-1} -60y_{n-2}+16y_{n-3}}{17} = \frac{h^4}{EI}f(x_{n-1}) \\
&\frac{72y_n -156y_{n-1} +96y_{n-2}-12y_{n-3}}{17} = \frac{h^4}{EI}f(x_{n})\\
\end{align*}

Dette kan skrives mer kompakt som en matriseligning:
\begin{equation}
\begin{bmatrix}
16	& -9	& \frac{8}{3}	& -\frac{1}{4}	&	&	&	&	&\\
-4	& 6		& -4			& 1				&	&	&	&	&\\
1	& -4	& 6				& -4			& 1	&	&	&	&\\
	& 1		& -4			& 6				&-4	& 1	&	&	&\\
	&  & \ddots		&\ddots			&\ddots	&\ddots	&\ddots	&\\
	&		& 1	& -4	& 6				& -4			& 1	&\\
	&		&	& 1	& -4	& 6				& -4			& 1\\
	&	&	&	&	&\frac{-12}{17}	&\frac{96}{17}	&\frac{-156}{17}	&\frac{72}{17}\\
	&	&	&	&	&\frac{16}{17}	&\frac{-60}{17}	&\frac{72}{17}	&\frac{-28}{17}\\
\end{bmatrix}
\begin{bmatrix}
y_1\\
y_2\\
\vdots\\
\\
\vdots\\
\\
\vdots\\
y_{n-1}\\
y_n\\
\end{bmatrix} =
\begin{bmatrix}
f(x_1)\\
f(x_2)\\
\vdots\\
\\
\vdots\\
\\
\vdots\\
f(x_{n-1})\\
f(x_n)\\
\end{bmatrix}
\end{equation}