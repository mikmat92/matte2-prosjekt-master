\section{Oppgaver}
\subsection{Oppgave 1}\label{sec:oppg1}
\subsubsection{5.1.21}
Vi benytter oss av Taylor's teorem:
\begin{quote}
Dersom \textit{f(x)} er \textit{k} ganger deriverbar på intervallet [\textit{x}, \textit{x\textsubscript{0}}], finnes det et tall \textit{c} slik at
\begin{equation} \label{eq:taylor}
f(x) = \sum_{n=0}^k	(\frac{f^{(k)}(x_{0}}{n!} (x-x_{0})^{k}) + \frac{f^{(k+1)}(c)}{(k+1)!}(x-x_{0})^{(k+1)}
\end{equation}
\end{quote}

La \textit{x = x $\pm$ h} og \textit{x\textsubscript{0} = x}. Taylorpolynomet av 5. grad blir da
\begin{quote}
\begin{equation} \label{eq:x+h}
f(x+h) = f(x) + f'(x)h + \frac{f''(x)}{2}h^2 + \frac{f'''(x)}{6}h^3 + \frac{f^{(4)}(x)}{24}h^4 + \frac{f^{(5)}(x)}{120}h^5 + \frac{f^{(6)}(c)}{720}h^6
\end{equation}
\begin{equation} \label{eq:x-h}
f(x-h) = f(x) - f'(x)h + \frac{f''(x)}{2}h^2 - \frac{f'''(x)}{6}h^3 - \frac{f^{(4)}(x)}{24}h^4 - \frac{f^{(5)}(x)}{120}h^5 + \frac{f^{(6)}(c)}{720}h^6
\end{equation}
\end{quote}

Vi ønsker å kvitte oss med alle deriverte utenom fjerdederiverte, så vi evaluerer (\ref{eq:x+h}) + (\ref{eq:x-h}) og får
\begin{quote}
\begin{equation} \label{eq:x+-h-sum}
f(x+h)+f(x-h) = 2f(x) + f''(x)h^2 + \frac{f^{(4)}(x)}{12}h^4 + \frac{f^{(6)}(c)}{360}h^6
\end{equation}
\end{quote}

Nå ser vi at vo har kvittet oss med alle deriverte utenom fjerde og andre grad. Vi utvikler så Taylorpolynomet av 5. grad med \textit{x = x $\pm$ 2h} og \textit{x\textsubscript{0} = x}:

\begin{quote}
\begin{equation} \label{eq:x+2h}
f(x+2h) = f(x) + f'(x)2h + \frac{f''(x)}{2}4h^2 + \frac{f'''(x)}{6}8h^3 + \frac{f^{(4)}(x)}{24}16h^4 + \frac{f^{(5)}(x)}{120}32h^5 + \frac{f^{(6)}(c)}{720}64h^6
\end{equation}
\begin{equation} \label{eq:x-2h}
f(x-2h) = f(x) - f'(x)2h + \frac{f''(x)}{2}4h^2 - \frac{f'''(x)}{6}8h^3 + \frac{f^{(4)}(x)}{24}16h^4 - \frac{f^{(5)}(x)}{120}32h^5 + \frac{f^{(6)}(c)}{720}64h^6
\end{equation}
\end{quote}

På samme måte som vi fikk ligning (\ref{eq:x+-h-sum}), evaluerer vi (\ref{eq:x+2h})+(\ref{eq:x-2h}):
\begin{quote}
\begin{equation} \label{eq:x+-2h-sum}
f(x+2h)+f(x-2h) = 2f(x) + f''(x)4h^2 + \frac{f^{(4)}(x)}{12}16h^4 + \frac{f^{(6)}(c)}{360}64h^6
\end{equation}
\end{quote}

For å eliminere andrederivertleddet, evaluerer vi 4 $\cdot$ (\ref{eq:x+-h-sum}) - (\ref{eq:x+-2h-sum}):
\begin{quote}
\begin{equation} \label{eq:finalSum}
4f(x+h) + 4f(x-h) - f(x+2h) - f(x-2h) = 6f(x) - h^4f^{(4)}(x) - \frac{f^{(6)}(c)}{6}h^6
\end{equation}
\end{quote}

Vi løser så denne med hensyn på \textit{f\textsuperscript{(4)}(x)}:
\begin{quote}
\begin{equation}
f^{(4)}(x) = \frac{f(x+2h)-4f(x-h)+6f(x)-4f(x+h)+f(x+2h)-\frac{f^{(6)}(c)}{6}h^2}{h^4}
\end{equation}
\end{quote}

Ser at restleddet $\frac{f^{(6)}(c)}{6}h^2 \in O(h^2)$, dvs
\begin{quote}
\begin{equation} \label{eq:fjerdederivert}
f^{(4)}(x) = \frac{f(x+2h)-4f(x-h)+6f(x)-4f(x+h)+f(x+2h)}{h^4} + O(h^2)
\end{equation}
\end{quote}

\subsubsection{5.1.22a}

Skal vise at hvis $f(x)=f'(x)=0$, så er:
\begin{quote}
\begin{equation}
f^{(4)}(x+h) = \frac{16f(x+h)-9f(x+2h)+\frac{8}{3}f(x+3h)-\frac{1}{4}f(x+4h)}{h^4}=O(h^2)
\end{equation}
\end{quote}

Har fra oppgave 5.1.21 at:

\begin{quote}
\begin{equation}
f^{(4)}(x)=\frac{f(x-2h)-4f(x-h)+6f(x)-4f(x+h)+f(x+2h)}{h^4}+O(h^2)
\end{equation}
\end{quote}

Gjør om uttrykket ved å sette inn h, og får:

\begin{quote}
\begin{equation}
f^{(4)}(x+h)=\frac{f(x-2h+h)-4f(x-h+h)+6f(x+h)-4f(x+h+h)+f(x+2h+h)}{h^4}+O(h^2)
\end{equation}
\end{quote}
\begin{quote}
\begin{equation}
f^{(4)}(x+h)=\frac{f(x-h)-4f(x)+6f(x+h)-4f(x+2h)+f(x+3h)}{h^4}+O(h^2)
\end{equation}
\end{quote} 

Setter inn dette for $f^{(4)}(x+h)$ i likning 1, og får:

\begin{quote}
\begin{equation}
\frac{f(x-h)-4f(x)+6f(x+h)-4f(x+2h)+f(x+3h)}{h^4}+O(h^2)= \frac {16f(x+h)-9f(x+2h)+\frac{8}{3}f(x+3h)-\frac{1}{4}f(x+4h)}{h^4}=O(h^2) 
\end{equation}
\end{quote}

\begin{quote}
\begin{equation}
\frac{-4(x)+f(x+h)+6f(x+h)-16f(x+h)-4f(x+2h)+9f(x+2h)+f(x+3h)-\frac{8}{3}f(x+3h)-\frac{1}{4}f(x+4h)+O(h^2}{h^4}
\end{equation}
\end{quote}
\begin{quote}
\begin{equation}
\frac{-4(x)+f(x+h)-10f(x+h)+5f(x+2h)-\frac{5}{3}f(x+3h)+\frac{1}{4}f(x+4h)+O(h^2}{h^4}=O(h^2)
\end{equation}
\end{quote}
Vi har også fra 5.1.21 at hvis $f(x)=f'(x)=0$ så er:
\begin{quote}
\begin{equation}
f(x-h)-10f(x+h)+5f(x+2h)-\frac{5}{3}f(x+3h)+\frac{1}{4}f(x+4h)=O(h^6)
\end{equation}
\end{quote}

Ser at dette stemmer med det vi allerede har, setter inn og får:

\begin{quote}
\begin{equation}
\frac{-4(x)+O(h^6)+O(h^2)}{h^4}=O(h^2)
\end{equation}
\end{quote}

\begin{quote}
\begin{equation}
0+O(h^{(6-4)}+O(h^2)=O(h^2)
\end{equation}

\begin{equation}
O(h^2)=O(h^2)
\end{equation}
\end{quote}

\subsection{Oppgave 6 a}\label{sec:oppg6}
Vi skal legge til en sinusformet haug på bjelken. Det betyr at vi legger til en funksjon 
\begin{equation} \label{eq:sinuslast}
s(x) = -pgsin\frac{\pi}{L}x
\end{equation}
til kraftdelen til \textit{f(x)}

Vi skal vise at \begin{equation}
y(x) = \frac{f}{24EI}x^{2}(x^2-4Lx+6L^2)-\frac{gpL}{EI\pi}(\frac{L^3}{\pi^3}sin\frac{\pi}{L}x-\frac{x^3}{6}+\frac{Lx^2}{2}-\frac{L^2x}{\pi^2})
\end{equation}
tilfredsstiller Euler-Bernoulli-likningen og randbetingelsene for en bhelke som er festet i den ene enden og fri i den andre: \begin{equation} \label{eq:randbetingelserBjelke}
y(0)=y'(0)=y''(L)=y'''(L)=0
\end{equation}

Setter vi sammen ligning (\ref{eq:eulerbernoulli}) og ligning (\ref{eq:sinuslast}) 
Dette gir oss likningen 
\begin{equation} \label{eq:sinusLikning}
EIy''''=-pgsin\frac{\pi}{L}x
\end{equation}

For å finne \textit{y(x)} må vi integrere likning (\ref{eq:sinusLikning}) fire ganger. Dette gir oss:

\begin{equation}
EIy'''(x)=\frac{L}{\pi}pgcos(\frac{\pi}{L}x)+C_1
\end{equation}
\begin{equation}
EIy''(x)=\frac{L^2}{\pi^2}pgsin(\frac{\pi}{L}x)+C_1x+C_2
\end{equation}
\begin{equation}
EIy'(x)=-\frac{L^3}{\pi^3}pgcos(\frac{\pi}{L}x)+\frac{1}{2}C_1x^2+C_2x+C_3
\end{equation}
\begin{equation}
EIy(x)=-\frac{L^4}{\pi^4}pgsin(\frac{\pi}{L}x)+\frac{1}{3}C_1x^3+\frac{1}{2}C_2x^2+C_3x+C_4)
\end{equation}

Ut fra randbetingelsene i ligning (\ref{eq:randbetingelserBjelke}) gir dette oss følgende:

\begin{equation}
y''''(L)=0 \rightarrow \frac{L}{EI\pi}pgcos(\pi)+C_1 \rightarrow \frac{L}{EI\pi}pg=C_1
\end{equation}

\begin{equation}
y'''(L)=0 \rightarrow \frac{L^2}{\pi^2}pgsin(\pi)+C_1x+C_2 \rightarrow C_2=-\frac{pgL^2}{EI\pi}
\end{equation}

\begin{equation}
y''(0)=0 \rightarrow -\frac{L^3}{\pi^3}pgcos(\pi)+C_3 \rightarrow C_3=\frac{L^3pg}{\pi^3}
\end{equation}

\begin{equation}
y'(0)=0 \rightarrow C_4=0
\end{equation}

Dette gir oss følgende ligning for \textit{y(x)}:

\begin{equation}
y(x)=\frac{1}{EI}(-\frac{L^4}{\pi^4}pgsin(\frac{\pi}{L}x)+\frac{L}{6\pi}pgx^3-\frac{L^2}{2\pi}pgx^2+\frac{L^3}{\pi^3}pgx)
\end{equation}

Vi ser at alle leddene har \textit{pgL} og $\frac{1}{\pi}$ til felles så vi trekker dette på utsiden av parentesen. I tillegg vil vi ha \textit{y(x)} alene på venstre side. 

\begin{equation}
y(x)=-\frac{pgL}{EI\pi}(\frac{L^3}{\pi^3}sin(\frac{\pi}{L}x)-\frac{x^3}{6}+\frac{1}{2}Lx^2+\frac{L^2}{\pi^2}x)
\end{equation}

Denne delen legges til bak det vi har vist i oppgave 4 a) og vi får:

\begin{equation}
y(x) = \frac{f}{24EI}x^{2}(x^2-4Lx+6L^2)-\frac{gpL}{EI\pi}(\frac{L^3}{\pi^3}sin\frac{\pi}{L}x-\frac{x^3}{6}+\frac{Lx^2}{2}-\frac{L^2x}{\pi^2})
\end{equation}

Hvilket skulle vises.

\subsection{Oppgave 4 a}\label{sec:oppg4}

Den korrekte løsningen av likningen med konstant \textit{f(x)=f} er

\begin{equation}
y(x)=(\frac{f}{24EI})x^2(x^2-4Lx+6L^2)
\end{equation}

Dette skal vi vise ved å derivere 4 ganger.

Først skriver vi om likningen

\begin{equation}
y(x)=\frac{fx^2(x^2-4Lx+6L^2}{24EI}
\end{equation}

Løser opp parentesen og får:

\begin{equation}
y(x)=\frac{fx^4-4fLx^3+6fL^2x^2}{24EI}
\end{equation}

Så deriverer vi fortløpende:

\begin{equation}
y'(x)=\frac{4fx^3-12fLx^2+12fL^2x}{24EI} = \frac{4(fx^3-fLx^2+3fL^2x}{24EI}
\end{equation}
\begin{equation}
\frac{fx^3-3fLx^2+3fL^2x}{6EI}
\end{equation}

\begin{equation}
y''(x) = \frac{3fx^2-6fLx+3fL^2}{6EI} = \frac{3(fx^2-2fLx+fL^2}{2EI}
\end{equation}

\begin{equation}
y'''(x) = \frac{2fx-2fL}{2EI} = \frac{fx-fL}{EI}
\end{equation}

\begin{equation}
y''''(x) = \frac{f}{EI} = \frac{f(x)}{EI}
\end{equation}

Hvilket skulle vises.

\subsection{Oppgave 4 b}\label{sec:oppg4}
Som gitt i oppgaveteksten har utledningen av formlene for de deriverte brukt Taylors formel med feilledd:
\begin{equation}
\frac{y^{(6)}*(c)}{6!} * h^6
\end{equation}

Vi vet fra oppgave 4a følgende:
\begin{equation}
y''''(x) = \frac{f}{EI}
\end{equation}
\begin{equation}
y^{(5)}(x) = 0
\end{equation}
\begin{equation}
y^{(6)}(x) = 0
\end{equation}
Vi vet at både $y^{(5)} $ og $y^{(6)} $ er lik null, dermed er løsningen eksakt.

